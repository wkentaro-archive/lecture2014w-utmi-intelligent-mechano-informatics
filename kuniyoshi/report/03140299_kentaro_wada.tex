%        File: 03140299_kentaro_wada.tex
%     Created: Tue Jul 21 02:00 AM 2015 J
% Last Change: Tue Jul 21 02:00 AM 2015 J
%
\documentclass[10pt,twocolumn]{jarticle}

%%% Packages ----------------------------------------
% to input Japanese
\usepackage[japanese]{babel}

%% something
% \usepackage{ascmac}

% to be standard a4paper
\usepackage{geometry}
\geometry{a4paper, left=20mm, right=20mm, top=20mm, bottom=40mm}

% to insert figures
\usepackage[dvipdfmx]{graphicx}

%%% -------------------------------------------------


%%% Titles ------------------------------------------
\title{知能機械情報学レポート課題1}
\author{03-140299, 機械情報工学科4年, 和田健太郎}

\begin{document}
\maketitle
%%% -------------------------------------------------


%%% Bodies ------------------------------------------
\section{概要}
Hopfield型のニューラルネットワークによって,2種類の2値(+1/-1)画像を記憶させ,
元画像にノイズを加えた画像を初期値として想起させる.
想起性能を調べる実験として以下のようなものを行った.
\begin{itemize}
  \item 画像の種類を変化させる.
  \item 画像に対して加えるノイズ量を変化させる.
\end{itemize}

想起性能としては正解と類似度の全試行平均(類似度平均)と
元画像の完全再現割合(正答率)を用いる.
%%% -------------------------------------------------


%%% -------------------------------------------------
\section{画像の種類数による想起性能変化}
%%% -------------------------------------------------
画像の種類を2から6まで変化させ,想起性能の変化を調べた.


%%% -------------------------------------------------
\section{入力画像のノイズ量による想起性能変化}
%%% -------------------------------------------------
入力画像に加えるノイズ量を5から50\%まで変化させ,想起性能の変化を調べた.


%%% Bibliography ------------------------------------
\begin{thebibliography}{9}
  \bibitem{bib1} Samuel R.Buss,"Introduction to Inverse Kinematics with Jacobian Transpose,Pseudoinverse and Damped Least Squares methods"
\end{thebibliography}
%%% -------------------------------------------------


\end{document}
